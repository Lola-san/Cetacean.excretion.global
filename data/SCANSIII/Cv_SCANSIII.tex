% Options for packages loaded elsewhere
\PassOptionsToPackage{unicode}{hyperref}
\PassOptionsToPackage{hyphens}{url}
%
\documentclass[
]{article}
\usepackage{lmodern}
\usepackage{amssymb,amsmath}
\usepackage{ifxetex,ifluatex}
\ifnum 0\ifxetex 1\fi\ifluatex 1\fi=0 % if pdftex
  \usepackage[T1]{fontenc}
  \usepackage[utf8]{inputenc}
  \usepackage{textcomp} % provide euro and other symbols
\else % if luatex or xetex
  \usepackage{unicode-math}
  \defaultfontfeatures{Scale=MatchLowercase}
  \defaultfontfeatures[\rmfamily]{Ligatures=TeX,Scale=1}
\fi
% Use upquote if available, for straight quotes in verbatim environments
\IfFileExists{upquote.sty}{\usepackage{upquote}}{}
\IfFileExists{microtype.sty}{% use microtype if available
  \usepackage[]{microtype}
  \UseMicrotypeSet[protrusion]{basicmath} % disable protrusion for tt fonts
}{}
\makeatletter
\@ifundefined{KOMAClassName}{% if non-KOMA class
  \IfFileExists{parskip.sty}{%
    \usepackage{parskip}
  }{% else
    \setlength{\parindent}{0pt}
    \setlength{\parskip}{6pt plus 2pt minus 1pt}}
}{% if KOMA class
  \KOMAoptions{parskip=half}}
\makeatother
\usepackage{xcolor}
\IfFileExists{xurl.sty}{\usepackage{xurl}}{} % add URL line breaks if available
\IfFileExists{bookmark.sty}{\usepackage{bookmark}}{\usepackage{hyperref}}
\hypersetup{
  pdftitle={CV computations SCANS III},
  pdfauthor={Lola Gilbert},
  hidelinks,
  pdfcreator={LaTeX via pandoc}}
\urlstyle{same} % disable monospaced font for URLs
\usepackage[margin=1in]{geometry}
\usepackage{longtable,booktabs}
% Correct order of tables after \paragraph or \subparagraph
\usepackage{etoolbox}
\makeatletter
\patchcmd\longtable{\par}{\if@noskipsec\mbox{}\fi\par}{}{}
\makeatother
% Allow footnotes in longtable head/foot
\IfFileExists{footnotehyper.sty}{\usepackage{footnotehyper}}{\usepackage{footnote}}
\makesavenoteenv{longtable}
\usepackage{graphicx,grffile}
\makeatletter
\def\maxwidth{\ifdim\Gin@nat@width>\linewidth\linewidth\else\Gin@nat@width\fi}
\def\maxheight{\ifdim\Gin@nat@height>\textheight\textheight\else\Gin@nat@height\fi}
\makeatother
% Scale images if necessary, so that they will not overflow the page
% margins by default, and it is still possible to overwrite the defaults
% using explicit options in \includegraphics[width, height, ...]{}
\setkeys{Gin}{width=\maxwidth,height=\maxheight,keepaspectratio}
% Set default figure placement to htbp
\makeatletter
\def\fps@figure{htbp}
\makeatother
\setlength{\emergencystretch}{3em} % prevent overfull lines
\providecommand{\tightlist}{%
  \setlength{\itemsep}{0pt}\setlength{\parskip}{0pt}}
\setcounter{secnumdepth}{-\maxdimen} % remove section numbering

\title{CV computations SCANS III}
\author{Lola Gilbert}
\date{25/03/2021}

\begin{document}
\maketitle

Our goal was to use abundances and CVs for specific sets of blocks of
SCANS III. We used result tables from SCANS III report : Hammond et
al.~(2017), Estimates of cetacean abundance in European Atlantic waters
in summer 2016 from the SCANS-III aerial and shipboard survey, and from
these we needed to compute the total abundances and CVs for our blocks
of interest. For instance, we want to consider one block ``A'' including
sub-blocks from SCANS III AA, AB and AC.

Our calculations were the followings:

\begin{itemize}
\tightlist
\item
  for block i, we have \(A_i\) the abundance and \(CV_i\) the
  coefficient of variation, from which we can compute the variance
  \(\sigma_i^2\) : \[ \sigma_i^2 = (A_i\times CV_i)^2 \]
\item
  for all blocks, the total abundance is the sum of abundances :
  \[ A_{tot} = \sum_i A_i \] And the total variance is the sum of
  variances : \[ \sigma_{tot}^2 = \sum_i \sigma_i^2\] From this, we can
  compute the total coefficient of variation :
  \[ CV_{tot} = \frac{\sigma_{tot}} {A_{tot}}\]
\end{itemize}

Before doing this on our blocks of interest, we wanted to verify our
calculations by checking we were getting the same total CVs for all
blocks as those indicated in the result tables of SCANS III. However, it
does not work for all species. For a few, we get very different values
(e.g.~harbour porpoise for aerial survey, we find CV = 0.0787 where in
the report it is 0.172) ; for others, we get close values, but still
different (e.g.~Risso's dolphins we find 0.505 where it is 0.513) and
finally, for a few other, we get values so close that it could be a
matter of significant numbers and rounding (e.g.~pilot whales for aerial
surveys, we get 0.606 where it is 0.605 in the report and undetermined
common or striped dolphins where we get 0.400 for 0.404 in the report.
Here is a summary of differences we get :

\begin{longtable}[]{@{}lccc@{}}
\toprule
Survey & Species & Our total CV value & total CV value from SCANS III
report\tabularnewline
\midrule
\endhead
aerial & harbour porpoise & \textcolor{red}{0.0787} &
0.172\tabularnewline
aerial & bottlenose dolphin & 0.217 & 0.242\tabularnewline
aerial & Risso's dolphin & 0.505 & 0.513\tabularnewline
aerial & white-beaked dolphin & 0.261 & 0.288\tabularnewline
aerial & white-sided dolphin & 0.683 & 0.704\tabularnewline
aerial & common dolphin & 0.150 & 0.188\tabularnewline
aerial & striped dolphin & 0.399 & 0.403\tabularnewline
aerial & Und common or striped dolphin & 0.170 & 0.201\tabularnewline
aerial & pilot whale & 0.606 & 0.605\tabularnewline
aerial & beaked whales & 0.305 & 0.376\tabularnewline
aerial & minke whale & 0.217 & 0.345\tabularnewline
ship & bottlenose dolphin & 0.463 & 0.505\tabularnewline
ship & common dolphin & 0.555 & 0.549\tabularnewline
ship & striped dolphin & 0.324 & 0.319\tabularnewline
ship & Und common or striped dolphin & 0.400 & 0.404\tabularnewline
ship & pilot whale & 0.347 & 0.389\tabularnewline
ship & beaked whales & 0.392 & 0.545\tabularnewline
ship & fin whale & 0.178 & 0.384\tabularnewline
ship & sperm whale & 0.776 & 0.402\tabularnewline
\bottomrule
\end{longtable}

\end{document}
